% Anexo 1

\chapter{Função de ambiguidade para um pulso FM} %

\label{Annex1} % para referenciares o anexo em alguma parte da tese usa o comando \ref{Annex1}

Código sugerido pelo MATLAB para criar função de ambiguidade para um pulso FM consoante as caraterísticas do sinal:

\begin{verbatim}
clc
close all
clear all

Rmax = 15e3;
Rres = 1500;
c = 3e8;
prf = c/(2*Rmax);
bw = c/(2*Rres);
fs = 2*bw;
fc = 1e9;

lfmwaveform = phased.LinearFMWaveform('SampleRate',fs,'SweepBandwidth',bw,'PRF',prf,'PulseWidth',5/bw);
bw_lfm = bandwidth(lfmwaveform);
wav = lfmwaveform();
deltav_lfm = dop2speed(20e3,c/fc);
[afmag_lfm,delay_lfm,doppler_lfm] = ambgfun(wav,lfmwaveform.SampleRate,lfmwaveform.PRF);

subplot(2,1,2)  
surf(delay_lfm*1e6,doppler_lfm/1e3,afmag_lfm,'LineStyle','none'); 
axis tight; grid on; view([140,35]); colorbar;
xlabel('Delay \tau (us)');ylabel('Doppler f_d (kHz)');
title('FM Pulse Waveform Ambiguity Function');
\end{verbatim}