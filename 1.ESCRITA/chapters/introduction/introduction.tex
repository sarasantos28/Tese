\begin{introduction}

%Escreva aqui a introdução

A introdução prepara o leitor para uma leitura organizada e uma melhor compreensão do trabalho científico que se está a apresentar. Deve por isso começar por apresentar, de forma breve, a problemática em que se insere o trabalho científico.

Sempre que for necessário fazer este enquadramento de forma detalhada e mais longa, deve apenas referir-se o assunto na introdução indicando que a apresentação mais detalhada será feita num dos primeiros capítulos (normalmente o primeiro).

Feita, contudo, esta apresentação do assunto, expondo a problemática subjacente ao mesmo, deve apresentar-se o problema, ou problemas, objeto do estudo efetuado, a que se segue uma explicação das vias seguidas na investigação e da forma como elas transparecem na estrutura adotada para a apresentação. 

Seguem-se, se forem relevantes e necessárias, algumas explicações sobre a metodologia do trabalho, terminando com os objetivos que se pretende alcançar. 

Sempre que for necessário, para uma melhor compreensão por parte do leitor, pode dividir-se a introdução em pequenos subcapítulos, indicando objetivos, vias seguidas na investigação, metodologias e resultados pretendidos.

\end{introduction}